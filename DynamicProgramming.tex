\ifx\PREAMBLE\undefined
\documentclass{report}
\usepackage[format = hang, font = bf]{caption}
\usepackage{subcaption}
% The following is needed in order to make the code compatible
% with both latex/dvips and pdflatex. Added for using UML generated by MetaUML.
\ifx\pdftexversion\undefined
\usepackage[dvips]{graphicx}
\else
\usepackage[pdftex]{graphicx}
\DeclareGraphicsRule{*}{mps}{*}{}
\fi
\usepackage{array}
\usepackage{amsmath}
\usepackage{amsthm}
\usepackage{mathtools}
\usepackage{boxedminipage}
\usepackage{listings}
\usepackage{multicol}
\usepackage{makecell}%diagonal line in table
\usepackage{float}%allowing forceful figure[H]
\usepackage{xcolor}
\usepackage{amsfonts}%allowing \mathbb{R}
\usepackage{amssymb}
\usepackage{alltt}
\usepackage{algorithmicx}
\usepackage[chapter]{algorithm} 
%chapter option ensures that algorithms are numbered within each chapter rather than in the whole article
\usepackage[noend]{algpseudocode} %If end if, end procdeure, etc is expected to appear, remove the noend option
\usepackage{xspace}
\usepackage{physics}
\usepackage{color}
\usepackage{tikz}
\usetikzlibrary{shapes,positioning}
\usepackage{url}
\def\UrlBreaks{\do\A\do\B\do\C\do\D\do\E\do\F\do\G\do\H\do\I\do\J\do\K\do\L\do\M\do\N\do\O\do\P\do\Q\do\R\do\S\do\T\do\U\do\V\do\W\do\X\do\Y\do\Z\do\[\do\\\do\]\do\^\do\_\do\`\do\a\do\b\do\c\do\d\do\e\do\f\do\g\do\h\do\i\do\j\do\k\do\l\do\m\do\n\do\o\do\p\do\q\do\r\do\s\do\t\do\u\do\v\do\w\do\x\do\y\do\z\do\0\do\1\do\2\do\3\do\4\do\5\do\6\do\7\do\8\do\9\do\.\do\@\do\\\do\/\do\!\do\_\do\|\do\;\do\>\do\]\do\)\do\,\do\?\do\'\do+\do\=\do\#\do\-}
\usepackage{xr}%allow cross-file references
\usepackage[breaklinks = true]{hyperref}
\lstset{
language = C++, 
showspaces = false,
breaklines = true, 
tabsize = 2, 
numbers = left, 
extendedchars = false, 
basicstyle = {\ttfamily \footnotesize}, 
keywordstyle=\color{blue!70}, 
commentstyle=\color{gray}, 
frame=shadowbox, 
rulesepcolor=\color{red!20!green!20!blue!20}, 
numberstyle={\color[RGB]{0,192,192}}, 
moredelim=[is][\underbar]{_}{_}
}
\mathchardef\myhyphen="2D
% switch-case environment definitions
\algblock{switch}{endswitch} 
\algblock{case}{endcase}
%\algrenewtext{endswitch}{\textbf{end switch}} %If end switch is expected to appear, uncomment this line.
\algtext*{endswitch} % Make end switch disappear
\algtext*{endcase}
\algnewcommand\algorithmicinput{\textbf{Input}}
\algnewcommand\Input{\item[\algorithmicinput]}
\algnewcommand\algorithmicoutput{\textbf{Output}}
\algnewcommand\Output{\item[\algorithmicoutput]}
\algnewcommand\algorithmicinputoutput{\textbf{input and output:}}
\algnewcommand\InputOutput{\item[\algorithmicinputoutput]}
\allowdisplaybreaks
\newtheorem{theorem}{Theorem}
\newtheorem{corollary}[theorem]{Corollary}
\newtheorem{lemma}[theorem]{Lemma}
\newtheorem{definition}{Definition}
\begin{document}
\fi
\chapter{Dynamic Programming}
In this chapter we will introduce the last algorithm design paradigm: dynamic programming.
\section{Max-weight Independent Sets}
Our first example of dynamic programming is a relatively simple graph problem.
\begin{description}
\item[Input]A path graph $G(V,E)$ with non-negative weights on vertices.
\item[Output]An independent set, i.e. a subset of $V$ in which no vertices are adjacent, of maximum total weight.
\end{description}
\begin{center}
\begin{tikzpicture}
\tikzstyle{chosen} = [fill=red!20!, circle, draw=red]
\tikzstyle{normal} = [draw, circle]
\node[normal] (0) at (-4,0) {1};
\node[chosen] (1) at (-2,0) {4};
\node[normal] (2) at (0,0) 	{5};
\node[chosen] (3) at (2,0)	{4};
\draw (0) -- (1) -- (2) --(3);
\end{tikzpicture}
\end{center}
In the example above, the WIS is obviously the two red nodes. Generally, a brute-force approach takes exponential time. An intuitive greedy algorithm does not guarantee a correct answer: it is actually wrong for the simple example above.  The divide-and-conquer paradigm cannot be applied because there is no natural correct way to combine solutions to the two sub-problems. This is when dynamic programming comes to our rescue.

Let's consider the structure of an optimal solution in terms of its relationship with solutions to smaller problems. Let $S\subseteq V$ be a max-weight independent set (IS) of $G$, $v_n$ be the last vertex of the path and $v_{n-1}$ be the last but one vertex. Denote $G$ with $v_n$ deleted as $G'$, and $G$ with $v_n,v_{n-1}$ deleted as $G''$.  
\begin{itemize}
\item If $v_n\notin S$, then $S$ must also be a max-weight IS of $G'$, which can be proved easily by contradiction.
\item If $v_n\in S$, then $v_{n-1}\notin S$. It can be proved easily by contradiction that $S-\{v_n\}$ is a max-weight IS of $G''$. 
\end{itemize}
Therefore, a max-weight IS of $G$ is either a max-weight IS of $G'$, or a max-weight IS of $G''$ + $v_n$. The same reasoning holds for smaller problems, which induces a correct recursive algorithm:
\begin{enumerate}
\item Recursively compute $S_1$ = max-weight IS of $G'$.
\item Recursively compute $S_2$ = max-weight IS of $G''$.
\item Return $S_1$ or $S_2\cup\{v_n\}$, whichever is better.
\end{enumerate}
The correctness of the algorithm can be verified by induction. However it takes exponential time because it is per se a variant of the brute-force algorithm.
\begin{center}
\begin{tikzpicture}[
level 1/.style={sibling distance=6cm, level distance=1.5cm},
level 2/.style={sibling distance=3cm, level distance=1.5cm},
level 3/.style={sibling distance=2cm, level distance=1.5cm}]
level 4/.style={sibling distance=1cm, level distance=1.5cm}]
\tikzstyle{node} = [draw, circle,minimum size=0.8cm]
\node[node] (0) at (0,0) {n}
child {node[node]{n-1}
	child {node[node]{n-2}
		child {node[node]{n-3}}
		child {node[node]{n-4}}
		}
	child {node[node]{n-3}
		child {node[node]{n-4}}
		child {node[node]{n-5}}
		}
	}
child {node[node]{n-2}
	child {node[node]{n-3}
		child {node[node]{n-4}}
		child {node[node]{n-5}}
		}
	child {node[node]{n-4}
		child {node[node]{n-5}}
		child {node[node]{n-6}}
		}
	};
\end{tikzpicture}
\end{center}
As shown above, each sub-problem is calculated multiple times. The number of distinct sub-problems is actually $O(n)$. If we can reformulate the recursive algorithm into a bottom-up iterative algorithm, and cache the solution to a sub-problem the first time it is solved, the problem can be solved in linear time, as shown in Algorithm \ref{maxweightis}.
\begin{algorithm}[ht]
\caption{Max-weight Independent Set(DP)}\label{maxweightis}
\begin{algorithmic}[1]
\Input{Path graph $G=(V,E)$ with non-negative weight $w_i$ for each vertex $v_i$. Sub graph-composed of the first $i$ vertices is denoted by $G_i$.}
\Output{Array $A$ with $A[i]=$ total weight of max-weight IS of $G_i$.}
\State{$A[0]=0,A[1]=w_1$.}
\For{$i = 2,3,\dots,n$}
\State{$A[i]=\max\{A[i-1], A[i-2]+w_i\}$}
\EndFor
\end{algorithmic}
\end{algorithm}
Algorithm \ref{maxweightis} only outputs the total weight of the max-weight IS of $G$. The IS itself can be reconstructed according to array $A$, as shown in Algorithm \ref{reconstructionis}. The running time is also $O(n)$.

\begin{algorithm}[ht]
\caption{Reconstruction of Max-weight Independent Set}\label{reconstructionis}
\begin{algorithmic}[1]
\Input{Array $A$ computed in Algorithm \ref{maxweightis}.}
\Output{The max-weight IS $S$ of path graph $G$.}
\State{Initialize $S=\emptyset$, $i=n$.}
\While{$i\geq 2$}
\If{$A[i-1]<A[i-2]+w_i$}
\State{Add $v_i$ to $S$}
\State{$i = i - 2$}
\Else
\State{$i = i - 1$}
\EndIf
\EndWhile
\If{$v_2\notin S$}
\State{Add $v_1$ to $S$}
\EndIf
\end{algorithmic}
\end{algorithm}
\section{Principles of DP}
After a concrete example, let's introduce a few general principles of dynamic programming. Typical DP problems  share the following traits:
\begin{enumerate}
\item It is easy to identify a small number of sub-problems. In the max-weight IS problem, the sub-problems are the max-weight IS of $G_i$ for $i=0,1,\dots,n$. The number of sub-problems is not necessarily linear, but it has to be reasonably small.
\item Given solutions to smaller sub-problems, larger sub-problems can be solved quickly can correctly. This is usually expressed as a recursive relation, for example $A[i]=\max\{A[i-1],A[i-2]+w_i\}$ in the max-weight IS problem.
\item The final solution can be computed quickly after solving all sub-problems. Usually it's just the answer to the largest sub-problem.
\end{enumerate}
\section{Knapsack Problem}
\begin{description}
\item[Input]$n$ items with non-negative value $v_i$ and non-negative integral size $w_i$ for item $i$. Capacity $W$, which is a non-negative integer.
\item[Output]A subset $S\subseteq\{1,2,\dots,n\}$ that maximizes $\sum\limits_{i\in S}v_i$ subject to the condition $\sum\limits_{i\in S}w_i\leq W$.
\end{description}
Again let's consider different situations for item $n$. If $n\notin S$, then $S$ must also be the optimal solution for the first $n-1$ items and capacity $W$. On the contrary, if $n\in S$, then $S-\{n\}$ must be the optimal solution for the first $n-1$ items and capacity $W-w_n$. Let $V_{i,x}$ represent the value of the best solution for the first $i$ items and capacity $x$, then recursively we have 
$$V_{i,x}=\max\{V_{i-1,x},V_{i-1,x-w_i}+v_i\}.$$
The sub-problems have been identified up to now: for each $i,x$ combination, there is a sub-problem. A DP algorithm is show in Algorithm \ref{knapsack}. The running time is obviously $O(nW)$.

\begin{algorithm}[ht]
\caption{Knapsack Problem(DP)}\label{knapsack}
\begin{algorithmic}[1]
\Input{$n$ items as stated above.}
\Input{$(n+1)\times(W+1)$ 2-D array $A$ with $A[i][x]=V_{i,x}$.}
\State{Initialize $A[0][x]=0$ for $x=0,1,\dots,W$.}
\For{$i=1,2,\dots,n$}
\For{$x=0,1,\dots,W$}
\If{$x\geq w_i$}
\State{$A[i][x]=\max\{A[i-1][x],A[i-1][x-w_i]+v_i\}$}
\Else\State{$A[i][x]=A[i-1][x]$}
\EndIf\EndFor\EndFor
\end{algorithmic}
\end{algorithm}

Array $A$ records the maximized sum of value for all sub-problems. The solution to the problem, i.e. the subset $S$, can be reconstructed according to $A$, as shown in Algorithm \ref{knapsackreconstruction}. Note that the running time is only $O(n)$.

\begin{algorithm}[ht]
\caption{Knapsack Reconstruction}\label{knapsackreconstruction}
\begin{algorithmic}[1]
\Input{Array $A$ computed in Algorithm \ref{knapsack}.}
\Output{Solution $S$ to the knapsack problem.}
\State{Initialize $S=\emptyset$, $i=n,x=W$.}
\While{$i\geq 1$}
\If{$A[i][x]\neq A[i-1][x]$}
\State{Add $i$ to $S$}
\State{$x=x-w_i$}
\EndIf
\State{$i=i-1$}
\EndWhile
\end{algorithmic}
\end{algorithm}

An example of Knapsack problem is show below. There are 4 items, and $W = 6$. 
\begin{center}
\begin{tabular}{c|cccc}
item  & 1 & 2 & 3 & 4\\\hline
$v_i$ & 3 & 2 & 4 & 4\\
$w_i$ & 4 & 3 & 2 & 3\\
\end{tabular}
\begin{tabular}{c|ccccc}
$A[i,x$]& $i=0$ & $i=1$ & $i=2$ & $i=3$ & $i=4$\\\hline
$x=0$   &	0   & 0     & 0     & 0     & 0\\
$x=1$   &	0   & 0     & 0     & 0     & 0\\
$x=2$   &	0   & 0     & 0     & 4     & 4\\
$x=3$   &	0   & 0     & 2     & 4     & 4\\
$x=4$   &	0   & 3     & 3     & 4     & 4\\
$x=5$   &	0   & 3     & 3     & 6     & 8\\
$x=6$   &	0   & 3     & 3     & 7     & 8\\
\end{tabular}
\end{center}

The maximum value is therefore 8, corresponding to the subset \{3,4\}.
\section{Sequence Alignment}
\begin{description}
\Input{String $X=x_1x_2\dots x_m$, $Y=y_1y_2\dots y_m$ over some alphabet $\Sigma$. Penalty $\alpha_{ab}$ for aligning $a$ with $b$, and $\alpha_{gap}$ for inserting a gap. Presumably $\alpha_{aa}=0,\forall a\in\Sigma$.}
\Output{An alignment of $X$ and $Y$ with minimum total penalty.}
\end{description}
Consider the last position of the alignment. There are 3 possible cases: $x_m\:\&\:y_n$, $x_m\:\&\:gap$, or $gap\:\&\:y_n$. Let $X'=X-x_m$ and $Y'=Y-y_j$. If the optimal alignment falls into the first case, then it can be proved by contradictory that it is the optimal alignment of $X'$ and $Y'$ plus aligning $x_m$ with $y_n$. Similar reasoning can be made for the other 2 cases. In general, let $X_i$ represent the first $i$ letters of $X$ and $Y_j$ represent the first $j$ letters of $Y$. Let $P_{ij}$ represent the optimal penalty for aligning $X_i$ and $Y_j$. Then we must have
\begin{equation*}
P_{ij}=\min\begin{cases}
\alpha_{x_iy_j}+P_{i-1,j-1}\\
\alpha_{gap}+P_{i,j-1}\\
\alpha_{gap}+P_{i-1,j}\\
\end{cases}\end{equation*}
As for the base cases, obviously we have $P_{0i}=P_{i0}=i\cdot\alpha_{gap}$. Now we hear the knock at the door of a DP algorithm, as shown in Algorithm \ref{alignmentdp}. Its running time is $O(mn)$.
\begin{algorithm}[ht]
\caption{Sequence Alignment(DP)}\label{alignmentdp}
\begin{algorithmic}[1]
\Input{Two strings $X,Y$ as stated above.}
\Output{$(m+1)\times(n+1)$ 2D array $A$ with $A[i][j]=P_{ij}$.}
\State{Initialize $A[i][0]=A[0][i]=i\cdot\alpha_{gap}$ for all $i$.}
\For{$i=1$ \textbf{to} $m$}
\For{$j=1$ \textbf{to} $n$}
\State{$A[i][j]=\min\{A[i-1][j-1]+\alpha_{ij},A[i][j-1]+\alpha_{gap},A[i-1][j]+\alpha_{gap}\}$}
\EndFor\EndFor
\end{algorithmic}
\end{algorithm}

Just like before, the actual solution can be reconstructed based on $A$, as shown in Algorithm \ref{alignmentreconstruction}. The running time is $O(m+n)$.
\begin{algorithm}[ht]
\caption{Sequence Alignment Reconstruction}\label{alignmentreconstruction}
\begin{algorithmic}[1]
\Input{Array $A$ computed in Algorithm \ref{alignmentdp}}.
\Output{The actual alignment}
\State{$i=m,j=n$}
\While{$i>0$ or $j>0$}
\If{$i==0$}
\State{Align all $j$ left characters in $Y$ align with a gap and return}
\ElsIf{$j==0$}
\State{Align all $i$ left characters in $X$ align with a gap and return}
\ElsIf{$A[i][j]==A[i-1][j-1]+\alpha_{ij}$}
\State{Align $x_i$ with $y_j$}
\State{$i=i-1,j=j-1$}
\ElsIf{$A[i][j]==A[i][j-1]+\alpha_{gap}$}
\State{Align $y_j$ with a gap}
\State{$j=j-1$}
\Else\State{Align $x_i$ with a gap}
\State{$i=i-1$}
\EndIf\EndWhile
\end{algorithmic}
\end{algorithm}
\section{Optimal Binary Search Trees}
\ifx\PREAMBLE\undefined
\end{document}
\fi